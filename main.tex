\documentclass[titlepage,landscape,a4paper,10pt]{article}
\usepackage{listings, color, fontspec, minted, setspace, titlesec, fancyhdr, dingbat, mdframed, multicol}
\usepackage{graphicx, amssymb, amsmath, textcomp}
\usepackage[Chinese]{ucharclasses}
\usepackage[left=1.5cm, right=0.7cm, top=1.7cm, bottom=0.0cm]{geometry}

%configure the top corners
\pagestyle{fancy}
\setlength{\headsep}{0.1cm}
\rhead{Page \thepage}
\lhead{上海交通大学 Shanghai Jiao Tong University}

%configure space between the two columns
\setlength{\columnsep}{30pt}

%configure fonts
\setmonofont{Consolas}[Scale=0.8]
\newfontfamily\substitutefont{SimHei}[Scale=0.8]
\setTransitionsForChinese{\begingroup\substitutefont}{\endgroup}

%configure minted to display codes 
\definecolor{Gray}{rgb}{0.9,0.9,0.9}

%remove leading numbers in table of contents
\setcounter{secnumdepth}{0}

%configure section style
%\titleformat{\section}
%	{\normalfont\normalsize}	% The style of the section title
%	{}					% a prefix
%	{0pt}				% How much space exists between the prefix and the title
%	{\quad}				% How the section is represented
\titleformat{\section}{\large}{}{0pt}{}
\titlespacing{\section}{0pt}{0pt}{0pt}

%enable section to start new page automatically
%\let\stdsection\section
%\renewcommand\section{\penalty-100\vfilneg\stdsection}

%\renewcommand\theFancyVerbLine{\arabic{FancyVerbLine}}
\renewcommand{\theFancyVerbLine}{\small \oldstylenums{\arabic{FancyVerbLine}}}

\setminted[cpp]{
	style=xcode,
	mathescape,
	linenos,
	autogobble,
	baselinestretch=1.0,
	tabsize=2,
	%bgcolor=Gray,
	frame=single,
	framesep=1mm,
	framerule=0.3pt,
	numbersep=1mm,
	breaklines=true,
	breaksymbolsepleft=2pt,
	%breaksymbolleft=\raisebox{0.8ex}{ \small\reflectbox{\carriagereturn}}, %not moe!
	%breaksymbolright=\small\carriagereturn,
	breakbytoken=false,
}
\setminted[java]{
	style=xcode,
	mathescape,
	linenos,
	autogobble,
	baselinestretch=1.0,
	tabsize=2,
	%bgcolor=Gray,
	frame=single,
	framesep=1mm,
	framerule=0.3pt,
	numbersep=1mm,
	breaklines=true,
	breaksymbolsepleft=2pt,
	%breaksymbolleft=\raisebox{0.8ex}{ \small\reflectbox{\carriagereturn}}, %not moe!
	%breaksymbolright=\small\carriagereturn,
	breakbytoken=false,
}
\setminted[text]{
	style=xcode,
	mathescape,
	linenos,
	autogobble,
	baselinestretch=1.0,
	tabsize=2,
	%bgcolor=Gray,
	frame=single,
	framesep=1mm,
	framerule=0.3pt,
	numbersep=1mm,
	breaklines=true,
	breaksymbolsepleft=2pt,
	%breaksymbolleft=\raisebox{0.8ex}{ \small\reflectbox{\carriagereturn}}, %not moe!
	%breaksymbolright=\small\carriagereturn,
	breakbytoken=false,
}

%configure titles
\title{\LARGE{Dracarys} \\[2ex] \Large{Team Referrence Library} }
\date{\today}

%THE SCL BEGINS
\begin{document}
\maketitle

\begin{multicols*}{2}

\begin{spacing}{0}
	\tableofcontents
\end{spacing}
\end{multicols*}

\begin{multicols}{2}

\newpage


\begin{spacing}{0.8}

\section{多边形与圆面积交}
\inputminted{cpp}{src/多边形与圆面积交.cpp}

\section{二维几何}
\inputminted{cpp}{merge/Geo2D.cpp}

\section{$n\log n$ 半平面交}
\inputminted{cpp}{merge/HalfPlaneIntersection.cpp}

\section{Delaunay 三角剖分}
\inputminted{cpp}{improve/DelaunayTriangulation.cpp}


\section{三维几何操作合并}
\inputminted{cpp}{merge/3DGeo.cpp}

\section{三维凸包}
\inputminted{cpp}{src/三维凸包.cpp}

\section{凸包上快速询问}
\inputminted{cpp}{improve/PlayWithConvex.cpp}

\section{圆的面积模板 ($n^2\log n$)}
\inputminted{cpp}{src/圆的面积模板.cpp}

\section{三角形的心}
\inputminted{cpp}{improve/Triangle.cpp}

\section{最小覆盖球}
\inputminted{cpp}{src/最小覆盖球.cpp}

\section{经纬度求球面最短距离}
\inputminted{cpp}{src/经纬度求球面最短距离.cpp}

\section{长方体表面两点最短距离}
\inputminted{cpp}{src/长方体表面两点最短距离.cpp}

\section{最大团}
\inputminted{cpp}{improve/MaximumClique.cpp}

\section{极大团计数}
\inputminted{cpp}{src/极大团计数.cpp}

\section{KM}
\inputminted{cpp}{imporve/Hungarian.cpp}

\section{最小树形图}
\inputminted{cpp}{imporve/LiuZhu.cpp}

\section{无向图最小割}
\inputminted{cpp}{src/无向图最小割.cpp}

\section{带花树}
\inputminted{cpp}{src/带花树.cpp}

\section{动态最小生成树}
\inputminted{cpp}{src/动态最小生成树.cpp}

\section{Hopcroft}
\inputminted{cpp}{src/Hopcroft.cpp}

\section{素数判定}
\inputminted{cpp}{src/素数判定.cpp}

\section{启发式分解}
\inputminted{cpp}{src/启发式分解.cpp}

\section{二次剩余}
\inputminted{cpp}{src/二次剩余.cpp}

\section{Pell 方程}
\inputminted{cpp}{src/Pell方程.cpp}

\section{蔡勒公式}
\inputminted{cpp}{src/蔡勒公式.cpp}

\section{Schreier-Sims}
\inputminted{cpp}{improve/SchreierSims.cpp}

\section{Romberg}
\inputminted{cpp}{src/Romberg.cpp}

\section{线性规划}
\inputminted{cpp}{src/线性规划.cpp}

\section{FFT}
\inputminted{cpp}{improve/FFT.cpp}

\section{NTT}
\inputminted{cpp}{improve/NTT.cpp}

\section{FWT}
\inputminted{cpp}{improve/FWT.cpp}

\section{回文串 manacher}
\inputminted{cpp}{src/回文串manacher.cpp}

\section{后缀数组 ( 倍增 )}
\inputminted{cpp}{src/后缀数组(nlogn).cpp}

\section{后缀数组 (DC3)}
\inputminted{cpp}{src/DC3.cpp}

\section{后缀自动机}
\inputminted{cpp}{src/后缀自动机.cpp}

\section{后缀树 (With Walk)}
\inputminted{improve/SuffixTree.cpp}

\section{后缀树 (With Pop Front)}
\inputminted{improve/SuffixTree2.cpp}

\section{字符串最小表示}
\inputminted{cpp}{src/字符串最小表示.cpp}

\section{轻重链剖分}
\inputminted{cpp}{src/轻重链剖分.cpp}

\section{KD Tree}
\inputminted{cpp}{src/KDTree.cpp}

\section{Splay Tree}
\inputminted{cpp}{src/Splay.cpp}

\section{Link Cut Tree}
\inputminted{cpp}{improve/LCT.cpp}

\section{Dominator Tree}
\inputminted{cpp}{improve/DominatorTree.cpp}

\section{DancingLinks}
\inputminted{cpp}{src/DancingLinks.cpp}
%
如果矩阵中所有的列均被删除 , 找到一组合法解 , 退出 \\
任意选择一个未被删除的列 c , \\
枚举一个未被删除的行 r, 且 Matrix[r][c]=1, 将 (r, c) 加入 Ans \\
枚举所有的列 j, Matrix[r][j]=1, 将第 j 列删除\\
枚举所有的行 i, Matrix[i][j]=1, 将第 i 行删除
%            AlgorithmX(Dep + 1)

%Procedure AlgorithmX(Dep)
%        如果h->right = h(即所有的列均被删除), 找到一组解, 退出.
%        利用h和right指针找到一个c, 满足size[c]最小.
%        如果size[c] = 0(当前列无法被覆盖), 无解, 退出.
%Cover(c)
%    for (i = c->down; i != c; i ← i->down)
%       for (j = i->right; j != i; j ← j->right) Cover(j->col)
%       将i结点加入Ans, AlgorithmX(Dep + 1)
%       for (j = i->left; j != i; j ← j->left) Recover(j->col)
%    	Recover(c)
%Soduku问题可以转化一个Exact Cover Problem:16 * 16 * 16行, (i, j, k)表示(i, j)这个格子填上字母k.16 * 16 * 4列分别表示第i行中的字母k, 第i列中的字母k, 第i个子矩阵中的字母k, 以及(i, j)这个格子.对于每个集合(i, j, k), 它包含了4个元素:Line(i, k), Col(j, k), Sub(P[i][j], k), Grid(i, j), 其中P[i][j]表示(i, j)这个格子所属的子矩阵.本题转化为一个4096行, 1024列, 且1的个数为16384个的矩阵.下面介绍解决一般的Exact Cover Problem的AlgorithmX.
%N皇后问题:关键是构建Exact Cover问题的矩阵:N * N行对应了N * N个格子, 6N-2列对应了N行, N列, 2N-1条主对角线, 2N-1条副对角线.第i行共4个1, 分别对应(i, j)这个格子所处的行, 列, 主对角线和副对角线.直接对这个矩阵作AlgorithmX是错误的, 虽然每行, 每列都恰好被覆盖一次, 但是对角线是最多覆盖一次, 它可以不被覆盖, 这与Exact Cover问题的定义是不同的.
%有两种处理的方法:
%1) 新增4N-2行, 每行只有一个1, 分别对应了2N-1条主对角线和2N-1条副对角线, 这样就可以保证某个对角线不被覆盖的时候, 可以使用新增行来覆盖.
%2) 每次选择一个size[]值最小的列c进行覆盖, 而这一步, 我们忽略掉所有的对角线列, 只考虑c为行和列的情况.
%事实证明, 第2)种方法的效果好很多, 因此这个问题可以使用AlgorithmX轻松得到解决.


\inputminted{cpp}{src/DancingLinks.cpp}


\section{弦图相关}
\begin{enumerate}
\item[1.] 团数 $\leq$ 色数 , 弦图团数 = 色数

\item[2.] 设 $next(v)$ 表示 $N(v)$ 中最前的点 . 
令 w* 表示所有满足 $A \in B$ 的 w 中最后的一个点 , 
判断 $v \cup N(v)$ 是否为极大团 , 
只需判断是否存在一个 w, 
满足 $Next(w)=v$ 且 $|N(v)| + 1 \leq |N(w)|$ 即可 . 

\item[3.] 最小染色 : 完美消除序列从后往前依次给每个点染色 , 
给每个点染上可以染的最小的颜色

\item[4.] 最大独立集 : 完美消除序列从前往后能选就选

\item[5.] 弦图最大独立集数 $=$ 最小团覆盖数 , 
最小团覆盖 : 
设最大独立集为 $\{p_1,p_2, \dots ,p_t\}$, 
则 $\{p_1\cup N(p_1), \dots , p_t \cup N(p_t)\}$ 
为最小团覆盖
\end{enumerate}


\section{图同构 Hash}

$$F_t(i) = 
    (F_{t-1}(i) \times A + 
    \sum_{i\rightarrow j} F_{t-1}(j) \times B + 
    \sum_{j\rightarrow i} F_{t-1}(j) \times C +
    D \times (i = a))\ mod\ P
$$

枚举点 a , 迭代 K 次后求得的就是 a 点所对应的 hash 值 

其中 K , A , B , C , D , P 为 hash 参数 , 可自选


%\section{魔方旋转群}
%\inputminted{cpp}{src/魔方旋转群.cpp}

\section{直线下有多少个格点}
\inputminted{cpp}{src/直线下格点统计.cpp}

\section{费用流}
\inputminted{cpp}{improve/MincostFlow.cpp}

\section{综合}
\subsection{弦图}
设 $next(v)$ 表示 $N(v)$ 中最前的点 . 
令 $w*$ 表示所有满足 $A \in B$ 的 $w$ 中最后的一个点 , 
判断 $v \cup N(v)$ 是否为极大团 , 
只需判断是否存在一个 $w \in w*$, 
满足 $Next(w)=v$ 且 $|N(v)| + 1 \leq |N(w)|$ 即可 . 
\subsection{五边形数}
$
    \prod_{n=1}^{\infty}{(1-x^{n})}=\sum_{n=0}^{\infty}{(-1)^{n}(1-x^{2n+1})x^{n(3n+1)/2}}
$
\subsection{重心}
半径为 $r$ , 圆心角为 $\theta$ 的扇形重心与圆心的距离为 $\frac{4r\sin(\theta/2)}{3\theta}$ \\
半径为 $r$ , 圆心角为 $\theta$ 的圆弧重心与圆心的距离为 $\frac{4r\sin^3(\theta/2)}{3(\theta-\sin(\theta))}$ \\
\subsection{第二类 Bernoulli number}
\begin{align*}
    B_m &= 1 - \sum_{k=0}^{m-1}{\binom{m}{k}\frac{B_{k}}{m-k+1}} \\
    S_m(n) &= \sum_{k=1}^{n}{k^{m}} = \frac{1}{m+1}\sum_{k=0}^{m}{\binom{m+1}{k}B_{k}n^{m+1-k}}
\end{align*}
\subsection{Stirling 数}
第一类 :n 个元素的项目分作 k 个环排列的方法数目\\
\begin{align*}
    s(n, k) &= (-1)^{n+k}|s(n, k)| \\
    |s(n, 0)| &=0\\ 
    |s(1, 1)| &=1 \\
    |s(n, k)| &=|s(n-1, k-1)|+(n-1)*|s(n-1, k)|
\end{align*}
第二类 :n 个元素的集定义 k 个等价类的方法数\\
\begin{align*}
    S(n,1)&=S(n,n)=1\\
    S(n,k)&=S(n-1,k-1)+k*S(n-1,k)
\end{align*}
\subsection{三角公式}

\begin{footnotesize}
\noindent
\mbox{\vbox to 11pt{  \hbox{$
\sin(a \pm b) = \sin a \cos b \pm \cos a \sin b
$}  }}
\
\mbox{\vbox to 11pt{  \hbox{$
\cos(a \pm b) = \cos a \cos b \mp \sin a \sin b
$}  }}
\\
\mbox{\vbox to 11pt{  \hbox{$
\tan(a \pm b) = \frac{\tan(a)\pm\tan(b)}{1 \mp \tan(a)\tan(b)}
$}  }}
\
\mbox{\vbox to 11pt{  \hbox{$
\tan(a) \pm \tan(b) = \frac{\sin(a \pm b)}{\cos(a)\cos(b)}
$}  }}
\\
\mbox{\vbox to 11pt{  \hbox{$
\sin(a) + \sin(b) = 2\sin(\frac{a + b}{2})\cos(\frac{a - b}{2})
$}  }}
\
\mbox{\vbox to 11pt{  \hbox{$
\sin(a) - \sin(b) = 2\cos(\frac{a + b}{2})\sin(\frac{a - b}{2})
$}  }}
\\
\mbox{\vbox to 11pt{  \hbox{$
\cos(a) + \cos(b) = 2\cos(\frac{a + b}{2})\cos(\frac{a - b}{2})
$}  }}
\
\mbox{\vbox to 11pt{  \hbox{$
\cos(a) - \cos(b) = -2\sin(\frac{a + b}{2})\sin(\frac{a - b}{2})
$}  }}
\\
\mbox{\vbox to 11pt{  \hbox{$
\sin(na) = n\cos^{n-1}a\sin a - \binom{n}{3}\cos^{n-3}a \sin^3a + \binom{n}{5}\cos^{n-5}a\sin^5a - \dots
$}  }}
\\
\mbox{\vbox to 11pt{  \hbox{$
\cos(na) = \cos^{n}a - \binom{n}{2}\cos^{n-2}a \sin^2a + \binom{n}{4}\cos^{n-4}a\sin^4a - \dots
$}  }}

\end{footnotesize}

%\inputminted{cpp}{src/综合.cpp}

%\section{基本形}
%\inputminted{cpp}{src/基本形.cpp}

%\section{树的计数}
%\inputminted{cpp}{src/树的计数.cpp}

%\section{代数}
%\inputminted{cpp}{src/代数.cpp}

%\section{三角公式}
%\inputminted{cpp}{src/三角公式.cpp}

\end{spacing}
\end{multicols}
\section{积分表}
\begin{footnotesize}
\fbox{
	\parbox{26cm}{
\mbox{\vbox to 11pt{  \hbox{\textbf {Integrals of Rational Functions} }  }}
\
\mbox{\vbox to 11pt{  \hbox{$
\int \frac{1}{1+x^2}dx = \tan^{-1}x
$}  }}
\
\mbox{\vbox to 11pt{  \hbox{$
\int \frac{1}{a^2+x^2}dx = \frac{1}{a}\tan^{-1}\frac{x}{a}
$}  }}
\
\mbox{\vbox to 11pt{  \hbox{$
\int \frac{x}{a^2+x^2}dx = \frac{1}{2}\ln|a^2+x^2|
$}  }}
\
\mbox{\vbox to 11pt{  \hbox{$
\int \frac{x^2}{a^2+x^2}dx = x-a\tan^{-1}\frac{x}{a}
$}  }}
\
\mbox{\vbox to 11pt{  \hbox{$
\int \frac{x^3}{a^2+x^2}dx = \frac{1}{2}x^2-\frac{1}{2}a^2\ln|a^2+x^2|
$}  }}
\
\mbox{\vbox to 11pt{  \hbox{$
\int \frac{1}{ax^2+bx+c}dx = \frac{2}{\sqrt{4ac-b^2}}\tan^{-1}\frac{2ax+b}{\sqrt{4ac-b^2}}
$}  }}
\
\mbox{\vbox to 11pt{  \hbox{$
\int \frac{1}{(x+a)(x+b)}dx = \frac{1}{b-a}\ln\frac{a+x}{b+x}, \text{ } a\ne b
$}  }}
\
\mbox{\vbox to 11pt{  \hbox{$
\int \frac{x}{(x+a)^2}dx = \frac{a}{a+x}+\ln |a+x|
$}  }}
\
\mbox{\vbox to 11pt{  \hbox{$
\int \frac{x}{ax^2+bx+c}dx  = \frac{1}{2a}\ln|ax^2+bx+c| \nonumber
\\ -\frac{b}{a\sqrt{4ac-b^2}}\tan^{-1}\frac{2ax+b}{\sqrt{4ac-b^2}}
$}  }}
\mbox{\vbox to 11pt{  \hbox{\textbf{Integrals with Roots} }  }}
\
\mbox{\vbox to 11pt{  \hbox{$
\int \frac{x}{\sqrt{x\pm a} } dx = \frac{2}{3}(x\mp 2a)\sqrt{x\pm a}
$}  }}
\
\mbox{\vbox to 11pt{  \hbox{$
\int \sqrt{\frac{x}{a-x}}dx  =  -\sqrt{x(a-x)}
%\nonumber \\ 
-a\tan^{-1}\frac{\sqrt{x(a-x)}}{x-a}
$}  }}
\
\mbox{\vbox to 11pt{  \hbox{$
\int \sqrt{\frac{x}{a+x}}dx  =  \sqrt{x(a+x)} 
%\nonumber \\  
-a\ln \left [ \sqrt{x} + \sqrt{x+a}\right] 
$}  }}
\
\mbox{\vbox to 11pt{  \hbox{$
\int  x \sqrt{x^2 \pm a^2} dx= \frac{1}{3}\left ( x^2 \pm a^2 \right)^{3/2} 
$}  }}
\
\mbox{\vbox to 11pt{  \hbox{$
\int  x \sqrt{ax + b}dx =
%\nonumber \\  
\frac{2}{15 a^2}(-2b^2+abx + 3 a^2 x^2)
\sqrt{ax+b}
$}  }}
\
\mbox{\vbox to 11pt{  \hbox{$
\int \sqrt{x(ax+b)} dx  = \frac{1}{4a^{3/2}}\left[(2ax + b)\sqrt{ax(ax+b)} \right. \nonumber
\\  \left.
-b^2 \ln \left| a\sqrt{x} + \sqrt{a(ax+b)} \right| \right ] 
$}  }}
\
\mbox{\vbox to 11pt{  \hbox{$
\int\sqrt{x^2 \pm a^2} dx  = \frac{1}{2}x\sqrt{x^2\pm a^2} 
%\nonumber \\ 
\pm\frac{1}{2}a^2 \ln \left | x + \sqrt{x^2\pm a^2} \right | 
$}  }}
\
\mbox{\vbox to 11pt{  \hbox{$
\int \sqrt{x^3(ax+b)} dx  =\left [ 
\frac{b}{12a}-
\frac{b^2}{8a^2x}+
\frac{x}{3}\right] 
\sqrt{x^3(ax+b)} \nonumber \\  + 
\frac{b^3}{8a^{5/2}}\ln \left | a\sqrt{x} + \sqrt{a(ax+b)} \right |
$}  }}
\
\mbox{\vbox to 11pt{  \hbox{$
\int  \sqrt{a^2 - x^2} dx  = \frac{1}{2} x \sqrt{a^2-x^2} 
%\nonumber \\  
+\frac{1}{2}a^2\tan^{-1}\frac{x}{\sqrt{a^2-x^2}}
$}  }}
\
\mbox{\vbox to 11pt{  \hbox{$
\int \frac{x^2}{\sqrt{x^2 \pm a^2}} dx  = \frac{1}{2}x\sqrt{x^2 \pm a^2}
%\nonumber \\  
\mp \frac{1}{2}a^2 \ln \left| x + \sqrt{x^2\pm a^2} \right | 
$}  }}
\
\mbox{\vbox to 11pt{  \hbox{$
\int \frac{1}{\sqrt{x^2 \pm a^2}} dx = \ln \left | x + \sqrt{x^2 \pm a^2} \right | 
$}  }}
\
\mbox{\vbox to 11pt{  \hbox{$
\int \frac{1}{\sqrt{a^2 - x^2}} dx = \sin^{-1}\frac{x}{a} 
$}  }}
\
\mbox{\vbox to 11pt{  \hbox{$
\int \frac{x}{\sqrt{x^2\pm a^2}}dx = \sqrt{x^2 \pm a^2} 
$}  }}
\
\mbox{\vbox to 11pt{  \hbox{$
\int \frac{x}{\sqrt{a^2-x^2}}dx = -\sqrt{a^2-x^2} 
$}  }}
\
\mbox{\vbox to 11pt{  \hbox{$
\int  \sqrt{a x^2 + b x + c} dx = 
\frac{b+2ax}{4a}\sqrt{ax^2+bx+c}
\nonumber \\  
+
\frac{4ac-b^2}{8a^{3/2}}\ln \left| 2ax + b + 2\sqrt{a(ax^2+bx^+c)}\right |
$}  }}
\
\mbox{\vbox to 11pt{  \hbox{$
\int  x \sqrt{a x^2 + bx + c} = \frac{1}{48a^{5/2}}\left ( 
2 \sqrt{a} \sqrt{ax^2+bx+c}
\right . \nonumber \\   
 \times \left( -3b^2 + 2 abx + 8 a(c+ax^2) \right)
 \nonumber \\   \left.
 + 3(b^3-4abc)\ln \left|b + 2ax + 2\sqrt{a}\sqrt{ax^2+bx+c} \right| \right)
$}  }}
\
\mbox{\vbox to 11pt{  \hbox{$
\int \frac{1}{\sqrt{ax^2+bx+c}}dx=
%==\nonumber \\ 
\frac{1}{\sqrt{a}}\ln \left| 2ax+b + 2 \sqrt{a(ax^2+bx+c)} \right | 
$}  }}
\
\mbox{\vbox to 11pt{  \hbox{$
\int  \frac{x}{\sqrt{ax^2+bx+c}}dx=
\frac{1}{a}\sqrt{ax^2+bx + c} \nonumber \\ 
-
\frac{b}{2a^{3/2}}\ln \left| 2ax+b + 2 \sqrt{a(ax^2+bx+c)} \right |
$}  }}
\
\mbox{\vbox to 11pt{  \hbox{$
\int\frac{dx}{(a^2+x^2)^{3/2}}=\frac{x}{a^2\sqrt{a^2+x^2}}
$}  }}
\
\mbox{\vbox to 11pt{  \hbox{\textbf{Integrals with Logarithms} }  }}
\
\mbox{\vbox to 11pt{  \hbox{$
\int \ln (ax + b) dx = \left ( x + \frac{b}{a} \right) \ln (ax+b) - x , a\ne 0
$}  }}
\
\mbox{\vbox to 11pt{  \hbox{$
\int \frac{\ln ax}{x} dx = \frac{1}{2}\left ( \ln ax \right)^2 
$}  }}
\
\mbox{\vbox to 11pt{  \hbox{$
\int \ln  ( x^2 + a^2 )\hspace{.5ex}\text{dx} = x \ln (x^2 + a^2  ) +2a\tan^{-1} \frac{x}{a} - 2x 
$}  }}
\
\mbox{\vbox to 11pt{  \hbox{$
\int \ln  ( x^2 - a^2 )\hspace{.5ex}\text{dx} = x \ln (x^2 - a^2  ) +a\ln \frac{x+a}{x-a} - 2x $}  }}
\
\mbox{\vbox to 11pt{  \hbox{$
\int x \ln (ax + b) dx  = \frac{bx}{2a}-\frac{1}{4}x^2 \nonumber
\\ 
+\frac{1}{2}\left(x^2-\frac{b^2}{a^2}\right)\ln (ax+b) 
$}  }}
\
\mbox{\vbox to 11pt{  \hbox{$
\int \ln   \left ( ax^2 + bx + c\right) dx  = \frac{1}{a}\sqrt{4ac-b^2}\tan^{-1}\frac{2ax+b}{\sqrt{4ac-b^2}}
\nonumber \\   -2x
 + \left( \frac{b}{2a}+x \right )\ln \left (ax^2+bx+c \right) 
$}  }}
\
\mbox{\vbox to 11pt{  \hbox{$
\int x \ln \left ( a^2 - b^2 x^2 \right ) dx  = -\frac{1}{2}x^2+ \nonumber
\\ 
\frac{1}{2}\left( x^2 - \frac{a^2}{b^2} \right ) \ln \left (a^2 -b^2 x^2 \right) 
$}  }}
\
\mbox{\vbox to 11pt{  \hbox{\textbf{Integrals with Exponentials} }  }}
\
\mbox{\vbox to 11pt{  \hbox{$
\int x^n e^{ax}\hspace{1pt}\text{d}x = \dfrac{x^n e^{ax}}{a} - 
\dfrac{n}{a}\int x^{n-1}e^{ax}\hspace{1pt}\text{d}x
$}  }} 
\
\mbox{\vbox to 11pt{  \hbox{$
\int x e^{-ax^2}\ \text{dx} = -\dfrac{1}{2a}e^{-ax^2} 
$}  }}
\
\mbox{\vbox to 11pt{  \hbox{\textbf {Integrals with Trigonometric Functions} }  }}
\
\mbox{\vbox to 11pt{  \hbox{$
\int \sin^3 ax dx = -\frac{3 \cos ax}{4a} + \frac{\cos 3ax} {12a} 
$}  }}
\
\mbox{\vbox to 11pt{  \hbox{$
\int \cos^2 ax dx = \frac{x}{2}+\frac{ \sin 2ax}{4a} 
$}  }}
\
\mbox{\vbox to 11pt{  \hbox{$
\int \cos^3 ax dx = \frac{3 \sin ax}{4a}+\frac{ \sin 3ax}{12a} 
$}  }}
\
\mbox{\vbox to 11pt{  \hbox{$
\int \cos ax \sin bx dx  = \frac{\cos[(a-b) x]}{2(a-b)} -
%\nonumber \\  
 \frac{\cos[(a+b)x]}{2(a+b)} , a\ne b
$}  }}
\
\mbox{\vbox to 11pt{  \hbox{$
\int \sin^2 ax \cos bx dx  = 
-\frac{\sin[(2a-b)x]}{4(2a-b)} \nonumber \\   
+ \frac{\sin bx}{2b} 
- \frac{\sin[(2a+b)x]}{4(2a+b)}
$}  }}
\
\mbox{\vbox to 11pt{  \hbox{$
\int \sin^2 x \cos x dx = \frac{1}{3} \sin^3 x
$}  }}
\
\mbox{\vbox to 11pt{  \hbox{$
\int \cos^2 ax \sin bx dx  = \frac{\cos[(2a-b)x]}{4(2a-b)} 
- \frac{\cos bx}{2b}
\nonumber \\  
 - \frac{\cos[(2a+b)x]}{4(2a+b)}
$}  }}
\
\mbox{\vbox to 11pt{  \hbox{$
\int \cos^2 ax \sin ax dx = -\frac{1}{3a}\cos^3{ax} 
$}  }}
\
\mbox{\vbox to 11pt{  \hbox{$
\int \sin^2 ax \cos^2 bx dx  = \frac{x}{4}
-\frac{\sin 2ax}{8a}-
\frac{\sin[2(a-b)x]}{16(a-b)}
\nonumber \\  
+\frac{\sin 2bx}{8b}-
\frac{\sin[2(a+b)x]}{16(a+b)}
$}  }}
\
\mbox{\vbox to 11pt{  \hbox{$
\int \sin^2 ax \cos^2 ax dx = \frac{x}{8}-\frac{\sin 4ax}{32a}
$}  }}
\
\mbox{\vbox to 11pt{  \hbox{$
\int \tan ax dx = -\frac{1}{a} \ln \cos ax 
$}  }}
\
\mbox{\vbox to 11pt{  \hbox{$
\int \tan^2 ax dx = -x + \frac{1}{a} \tan ax 
$}  }}
\
\mbox{\vbox to 11pt{  \hbox{$
\int \tan^3 ax dx = \frac{1}{a} \ln \cos ax + \frac{1}{2a}\sec^2 ax 
$}  }}
\
\mbox{\vbox to 11pt{  \hbox{$
\int \sec x dx  = \ln | \sec x + \tan x | = 2 \tanh^{-1} \left (\tan \frac{x}{2} \right) 
$}  }}
\
\mbox{\vbox to 11pt{  \hbox{$
\int \sec^2 ax dx = \frac{1}{a} \tan ax 
$}  }}
\
\mbox{\vbox to 11pt{  \hbox{$
\int \sec^3 x \hspace{2pt}\text{dx} = \frac{1}{2} \sec x \tan x + \frac{1}{2}\ln | \sec x + \tan x |
$}  }}
\
\mbox{\vbox to 11pt{  \hbox{$
\int \sec x \tan x dx = \sec x 
$}  }}
\
\mbox{\vbox to 11pt{  \hbox{$
\int \sec^2 x \tan x dx = \frac{1}{2} \sec^2 x 
$}  }}
\
\mbox{\vbox to 11pt{  \hbox{$
\int \sec^n x \tan x dx = \frac{1}{n} \sec^n x , n\ne 0
$}  }}
\
\mbox{\vbox to 11pt{  \hbox{$
\int \csc x dx = \ln \left | \tan \frac{x}{2} \right|  = \ln | \csc x - \cot x| + C
$}  }}
\
\mbox{\vbox to 11pt{  \hbox{$
\int \csc^2 ax dx = -\frac{1}{a} \cot ax 
$}  }}
\
\mbox{\vbox to 11pt{  \hbox{$
\int \csc^3 x dx = -\frac{1}{2}\cot x \csc x + \frac{1}{2} \ln | \csc x - \cot x | 
$}  }}
\
\mbox{\vbox to 11pt{  \hbox{$
\int \csc^nx \cot x dx = -\frac{1}{n}\csc^n x, n\ne 0
$}  }}
\
\mbox{\vbox to 11pt{  \hbox{$
\int \sec x \csc x dx = \ln | \tan x | 
$}  }}
\
\mbox{\vbox to 11pt{  \hbox{\textbf{Products of Trigonometric Functions and Monomials} }  }}
\
\mbox{\vbox to 11pt{  \hbox{$
\int x \cos x dx = \cos x + x \sin x 
$}  }}
\
\mbox{\vbox to 11pt{  \hbox{$
\int x \cos ax dx = \frac{1}{a^2} \cos ax + \frac{x}{a} \sin ax 
$}  }}
\
\mbox{\vbox to 11pt{  \hbox{$
\int x^2 \cos x dx = 2 x \cos x + \left ( x^2 - 2 \right ) \sin x 
$}  }}
\
\mbox{\vbox to 11pt{  \hbox{$
\int x^2 \cos ax dx = \frac{2 x \cos ax }{a^2} + \frac{ a^2 x^2 - 2  }{a^3} \sin ax 
$}  }}
\
\mbox{\vbox to 11pt{  \hbox{$
\int x \sin x dx = -x \cos x + \sin x 
$}  }}
\
\mbox{\vbox to 11pt{  \hbox{$
\int x \sin ax dx = -\frac{x \cos ax}{a} + \frac{\sin ax}{a^2} 
$}  }}
\
\mbox{\vbox to 11pt{  \hbox{$
\int x^2 \sin x dx = \left(2-x^2\right) \cos x + 2 x \sin x
$}  }}
\
\mbox{\vbox to 11pt{  \hbox{$
\int x^2 \sin ax dx =\frac{2-a^2x^2}{a^3}\cos ax +\frac{ 2 x \sin ax}{a^2} 
$}  }}
\
\mbox{\vbox to 11pt{  \hbox{\textbf{Products of Trigonometric Functions and Exponentials} }  }}
\
\mbox{\vbox to 11pt{  \hbox{$
\int e^x \sin x dx = \frac{1}{2}e^x (\sin x - \cos x) 
$}  }}
\
\mbox{\vbox to 11pt{  \hbox{$
\int e^{bx} \sin ax dx = \frac{1}{a^2+b^2}e^{bx} (b\sin ax - a\cos ax) 
$}  }}
\
\mbox{\vbox to 11pt{  \hbox{$
\int e^{bx} \cos ax dx = \frac{1}{a^2 + b^2} e^{bx} ( a \sin ax + b \cos ax ) 
$}  }}
\
\mbox{\vbox to 11pt{  \hbox{$
\int x e^x \sin x dx = \frac{1}{2}e^x (\cos x - x \cos x + x \sin x) 
$}  }}
\
\mbox{\vbox to 11pt{  \hbox{$
\int x e^x \cos x dx = \frac{1}{2}e^x (x \cos x 
- \sin x + x \sin x) 
$}  }}
\
\mbox{\vbox to 11pt{  \hbox{$
\int e^x \cos x dx = \frac{1}{2}e^x (\sin x + \cos x)  
$}  }}
\
  }
}
\end{footnotesize}



\begin{multicols}{2}
\begin{spacing}{0.8}

%\section{积分表}
%\inputminted{cpp}{src/积分表.cpp}

\section{Java}
\inputminted{java}{src/Main.java}

\section{Vimrc}
\inputminted{text}{src/vimrc.vim}
\end{spacing}

\end{multicols}
\end{document}
%THE SCL ENDS
